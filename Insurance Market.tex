\documentclass[10pt]{beamer}
\usepackage[utf8]{inputenc}
\usepackage{xeCJK}
\usepackage{graphicx}
\usepackage {mathtools}
\usepackage{utopia} %font utopia imported
\usetheme{CambridgeUS}
\usecolortheme{dolphin}

% set colors
\definecolor{myNewColorA}{RGB}{25,25,112}
\definecolor{myNewColorB}{RGB}{25,25,112}
\definecolor{myNewColorC}{RGB}{25,25,112}
\setbeamercolor*{palette primary}{bg=myNewColorC}
\setbeamercolor*{palette secondary}{bg=myNewColorB, fg = white}
\setbeamercolor*{palette tertiary}{bg=myNewColorA, fg = white}
\setbeamercolor*{titlelike}{fg=myNewColorA}
\setbeamercolor*{title}{bg=myNewColorA, fg = white}
\setbeamercolor*{item}{fg=myNewColorA}
\setbeamercolor*{caption name}{fg=myNewColorA}
\usefonttheme{professionalfonts}
\usepackage{natbib}
\usepackage{hyperref}
%------------------------------------------------------------

\setbeamerfont{title}{size=\large}
\setbeamerfont{subtitle}{size=\small}
\setbeamerfont{author}{size=\small}
\setbeamerfont{date}{size=\small}
\setbeamerfont{institute}{size=\small}
\title[NYU MSQE]{Matching}

\author[Junrui Lin]{Junrui Lin}

\institute[jl12680]{NYU MSQE}
\date[\textcolor{white}{\today} ]
{\today}

%------------------------------------------------------------
%This block of commands puts the table of contents at the 
%beginning of each section and highlights the current section:
%\AtBeginSection[]
%{
%  \begin{frame}
%    \frametitle{Contents}
%    \tableofcontents[currentsection]
%  \end{frame}
%}
\AtBeginSection[]{
  \begin{frame}
  \vfill
  \centering
  \begin{beamercolorbox}[sep=8pt,center,shadow=true,rounded=true]{title}
    \usebeamerfont{title}\insertsectionhead\par%
  \end{beamercolorbox}
  \vfill
  \end{frame}
}
%------------------------------------------------------------

\begin{document}

%The next statement creates the title page.
\frame{\titlepage}

%------------------------------------------------------------

\begin{frame}{Risk Preference Estimation (Cohen and Einav, 2007)}
\textbf{Setup}
\begin{itemize}
\item $x_i$: vector of characteristics individual $i$ reports to the insurance company
\item $p_{it} = f_t(x_i)$: regular premium function that unknown for us, stable over time
\item $d_{it} = \min\{\frac{1}{2}p_{it},\ cap_t\}$: regular deductible level
\item $w_i$: wealth of individual $i$
\item $(p_i^h,d_i^h)$: insurance contract, high premium, high deductible
\item $(p_i^l,d_i^l)$: insurance contract, low premium, low deductible
\item $t_i$: duration of the policy
\item $u_i(w)$: individual's vNM utility function
\item $\lambda_i$: claim rate, known to individual (assume no moral harzard)
\end{itemize}
\end{frame}

\begin{frame}{Risk Preference Estimation (Cohen and Einav, 2007)}
\textbf{Model}

The expected utility from a contract $(p,d)$ is
$$v(p,d):=(1-\lambda t)u(w-pt)+(\lambda t)u(w-pt-d)$$
when individual is indifference between $(p^h,d^h)$ and $(p^l,d^l)$
\begin{align*}
	\lambda &= \lim_{t\to 0}\frac{\frac{1}{t}\bigg(u(w-p^h t)-u(w-p^lt))\bigg)}{\bigg(u(w-p^h t)-u(w-p^ht-d^h)-u(w-p^lt)+u(w-p^lt-d^l)\bigg)}\\
	&=\frac{(p^l-p^h)u'(w)}{u(w-d^l)-u(w-d^h)}
\end{align*}
Using Taylor expansion 
$$\frac{p^l-p^h}{\lambda}u'(w) \approx (d^h-d^l) u'(w)-\frac{1}{2}(d^h-d^l)(d^h+d^l)u''(w)$$
\end{frame}

\begin{frame}{Risk Preference Estimation (Cohen and Einav, 2007)}
\textbf{Model} (cont.)

Let $\Delta d = d^h-d^l$, $\Delta p = p^l-p^h$, and $\bar{d} = \frac{1}{2}(d^h+d^l)$, then previous Taylor expansion will be
$$\frac{\Delta p}{\lambda}u'(w)=\Delta d u'(w)-\bar d \Delta d u''(w)\Rightarrow r:=\frac{-u''(w)}{u'(w)}=\frac{\frac{\Delta p}{\lambda\Delta d}-1}{\bar d}$$
where $r$ is coefficient of absolute risk aversion at wealth level $w$.

\bigskip
\textbf{Note}: the indifferent set $(r^*(\lambda),\lambda)$ and $(\lambda^*(r),r)$ are interchangeable and have closed form.
\end{frame}

\begin{frame}{Risk Preference Estimation (Cohen and Einav, 2007)}
\textbf{Model Estimation}

We want to estimate the joint distribution of $(\lambda_i, r_i)$, conditional on observables $x_i$. Assuming $(\lambda_i, r_i)$ is bivariate lognormal
\begin{align*}
	\ln \lambda_i = x_i'\beta+\varepsilon_i,\ \ln r_i = x_i'\gamma + \nu_i
\end{align*}
$$\begin{bmatrix}
	\varepsilon_i\\
	\nu_i
\end{bmatrix}\overset{\text{i.i.d.}}{\sim}N\bigg(\begin{bmatrix}
	0\\
	0
\end{bmatrix},\begin{bmatrix}
	\sigma_\lambda^2 & \rho\sigma_\lambda\sigma_r\\
	\rho\sigma_\lambda\sigma_r & \sigma_r^2
\end{bmatrix}\bigg)$$
\bigskip
$\lambda$ and $r$ are not directly observed, we only observe claims and deductible choice.

Assume claims follows Poisson process
$$claims_i \sim Poisson(\lambda_i t_i)$$
When choosing deductible plans, they will choose low deductible iff $r_i\geq r_i^*(\lambda_i)$
$$Pr(\text{low deductible})=Pr\bigg(\exp(x_i'\gamma+\nu_i)>\frac{\frac{\Delta p_i}{\exp(x_i'\beta+\varepsilon_i)\Delta d_i}-1}{\bar d_i}\bigg)$$
Then it's fairly easy to write out the likelihood or use simulation method
\end{frame}


\begin{frame}{Asymmetric Information (Einav et al. 2010)}
\textbf{Setup}

\end{frame}

\begin{frame}{Asymmetric Information (Einav et al. 2010)}


\end{frame}

\end{document}



